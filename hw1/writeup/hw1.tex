\documentclass[12pt]{article}
\setlength{\topmargin}{0in}
\setlength{\headheight}{0in}
\setlength{\headsep}{0in}
\setlength{\textheight}{8.7in}
\setlength{\textwidth}{6.5in}
\setlength{\oddsidemargin}{0in}
\setlength{\evensidemargin}{0in}
\setlength{\parindent}{0.15in}
\setlength{\parskip}{0.10in}

\usepackage{graphicx}
\usepackage{listings}


\begin{document}

\title{PHYS386: HW1}
\author{Miles Wu}
\maketitle

\section{Q1}
We chose a discrete time of $\Delta t = 1/1024~\textnormal{sec}$. The FFT (we have taken the absolute value of the complex values) has only two non-zero points at $f = \pm 1~\textnormal{Hz}$. This is because the FFT cannot tell the difference between a wave travelling in the positive direction or one travelling in the negative direction, so the FFT values are mirrored about $f=0$. The amplitude is also $0.5$ because the amplitude of the sine wave was $1$ and half is assigned to each peak. To get the $V_{RMS}$ we would need to square these values, add the two peaks together and then square root again (in other words $\sqrt{0.5^2+0.5^2} = 1/\sqrt{2}$.

\begin{center}
\includegraphics[width=3.5in]{q1.png}
\end{center}

\section{Q2}
We generated a random complex power spectrum ($N=1024$ and $\Delta T = 1~\textnormal{s}$) using two normal distributions with mean zero and variance one (making sure the $f=0$ bin is zero). This was then band-limited using an RC filter with $\tau = 5$ to avoid frequencies near the Nyquist frequency. This was then inverse FFT'd to obtain the time stream and then we compute the autocorrelation function. The integral of the filtered power spectrum was $0.000104~\textnormal{W}^2$ and the integral of the autocorrelation function was also $0.000104~\textnormal{W}^2$.

\begin{center}
\includegraphics[width=5.7in]{q2.png}
\end{center}

\section{Q3}
Although it is a little hard to see any patterns between the two channels due to the offsets between the levels, it seems like there is a lot of correlation between the two channels (in that both follow the same downward trend). Since it appears in two channels in the same way, it can't be the CMB signal nor can it be noise from the detector electronics (as we assume the electronics for each channel are independent). Instead we presume that it is thermal emissions noise from the atmosphere that affects all the channels in the same way at the same time.

\begin{center}
\includegraphics[width=4.5in]{q3a.png}
\end{center}
\begin{center}
\includegraphics[width=4.5in]{q3b.png}\\
Green is channel 2, blue is channel 1 
\end{center}


\section{Q4}
Here we used a Hanning windowing function and averaged the FFT spectrums from region where the scan flag mask was set to good (the regions were truncated to a power of 2 to make our lives easier). We also plotted linear power spectral density (units of $\textnormal{W}/\sqrt{\textnormal{Hz}}$).

For the cross spectrum between channel 1 and channel 2, we again plotted the linear power density (units of $\textnormal{W}/\sqrt{\textnormal{Hz}}$). However, for the phase plot we did not do the square root as this would halve all the complex arguments.
\begin{center}
\includegraphics[width=4.5in]{q4.png}
\end{center}

\section{Source Code}
\subsection{Q1}
\lstinputlisting[language=Python]{../q1.py}
\subsection{Q2}
\lstinputlisting[language=Python]{../q2.py}
\subsection{Q3/4}
\lstinputlisting[language=Python]{../q3.py}



\end{document}