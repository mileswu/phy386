\documentclass[12pt]{article}
\setlength{\topmargin}{0in}
\setlength{\headheight}{0in}
\setlength{\headsep}{0in}
\setlength{\textheight}{8.7in}
\setlength{\textwidth}{6.5in}
\setlength{\oddsidemargin}{0in}
\setlength{\evensidemargin}{0in}
\setlength{\parindent}{0.15in}
\setlength{\parskip}{0.10in}

\usepackage{graphicx}


\begin{document}

\title{PHYS386: HW1}
\author{Miles Wu}
\maketitle

\section{Q1}
We chose a discrete time of $\Delta t = 1/1024~\textnormal{sec}$. The FFT (we have taken the absolute value of the complex values) has only two non-zero points at $f = \pm 1~\textnormal{Hz}$. This is because the FFT cannot tell the difference between a wave travelling in the positive direction or one travelling in the negative direction, so the FFT values are mirrored about $f=0$. The amplitude is also $0.5$ because the amplitude of the sine wave was $1$ and half is assigned to each peak. To get the $V_{RMS}$ we would need to square these values, add the two peaks together and then square root again (in other words $\sqrt{0.5^2+0.5^2} = 1/\sqrt{2}$.

\begin{center}
\includegraphics[width=3.5in]{q1.png}
\end{center}

\section{Q2}

\section{Q3}
Although it is a little hard to see any patterns between the two channels due to the offsets between the levels, it seems like there is some correlation between the two channels.
EXPLAIN
\begin{center}
\includegraphics[width=4.5in]{q3a.png}
\end{center}
\begin{center}
\includegraphics[width=4.5in]{q3b.png}\\
Green is channel 2, blue is channel 1 
\end{center}


\section{Q4}
Here we used a Hanning windowing function and averaged the FFT spectrums from region where the scan flag mask was set to good (the regions were truncated to a power of 2 to make our lives easier). We also plotted linear power spectral density (units of $\textnormal{W}/\sqrt{\textnormal{Hz}}$).

For the cross spectrum between channel 1 and channel 2, we again plotted the linear power density (units of $\textnormal{W}/\sqrt{\textnormal{Hz}}$). However, for the phase plot we did not do the square root as this would halve all the complex arguments.
\begin{center}
\includegraphics[width=4.5in]{q4.png}
\end{center}


\end{document}