\documentclass[12pt]{article}
\setlength{\topmargin}{0in}
\setlength{\headheight}{0in}
\setlength{\headsep}{0in}
\setlength{\textheight}{8.7in}
\setlength{\textwidth}{6.5in}
\setlength{\oddsidemargin}{0in}
\setlength{\evensidemargin}{0in}
\setlength{\parindent}{0.15in}
\setlength{\parskip}{0.10in}

\usepackage{graphicx}
\usepackage{listings}


\begin{document}

\title{PHYS386: HW3}
\author{Miles Wu}
\maketitle

\section{Q1}
For each pixel in the image, we fit a linear line to the 88 reads and the slope of this line represents the current on that pixel.
To get an idea of the distribution of these slopes, we next make a histogram plot of the slopes:

\begin{center}
\includegraphics[width=4in]{hw3-q0-hist.png}
\end{center}

Most of the pixels are centered around a very small band at around $-0.5$, but there is a very long tail on the positive side.
This long tail is because of hot pixels that can have very large slopes.

We next plot the dark current for the first $512 \times 512$ pixels in the first quadrant, using $-1$ to $0$ as the range for our colour scale.
This range was manually specified so that we can see the dark current signal without the hot pixels throwing our range way off; the hot pixels are simply clipped in the image.

\begin{center}
\includegraphics[width=4.5in]{hw3-q0-darkcurrent.png}
\end{center}

\section{Q2}
We wish to next plot the power spectrum of the first quadrant, but the hot pixels might throw this calculation off.
We define a hot pixel to be one that has a slope greater than $0$ or one that has one less than $-1$.
For each hot pixel we replace its value with the median of the 10 pixels on either side of it in the row.
This removes the hot pixel's value without introducing a sudden discontinuity (which would happen if we replaced it with a global mean value or similar).
We chose to take the median, as the median wouldn't be as affected as a mean by hot pixel values, and we looked at 20 pixels just in case the immediate neighbour pixels are also hot.
\begin{center}
\includegraphics[width=4.5in]{hw3-q0-darkcurrent-filtered.png}
\end{center}

There is actually a gap of 12 timesteps between each row of the data, so we add in 12 zeros to the end of each row.
We next use a Hanning function on the data and do an FFT.
After normalizing and taking the absolute value, we plot this:
\begin{center}
\includegraphics[width=4in]{hw3-q0-ps.png}
\end{center}

There are large spikes at multiples of $512$ (e.g. 512, 1024, 1536 etc.).
This is because of the 12 zeros we added that introduce a signal that has period $512$.
We can simply ignore this artefact.

\section{Q3}
If we look at dark current of the second quadrant (flipped as it is readout backwards) we see there is a large amount of correlation between it and the first quadrant.

\begin{center}
\includegraphics[width=4.5in]{hw3-q1-darkcurrent-filtered.png}
\end{center}

To confirm this we plot the power cross-spectrum of quadrant 1 and quadrant 2:
\begin{center}
\includegraphics[width=4in]{hw3-q0q1-ps.png}
\end{center}

As we can see, it is very similar to the power spectrum of quadrant 1, indicating there is a lot of commonality between the two.
To remove this common noise that is in both quadrants we can again use PCA.
We find the two eigenvalues to be $0.3067$ and $0.00417$ and the primary eigenvector to be $(0.748, 0.663)$.
This is fairly close to the normalized diagonal vector $(1/\sqrt{2}, 1/\sqrt{2})$ which tells us that the largest variance is caused by the noise that affects both channels at once in roughly the same way.

To remove the noise, we subtract off the component of the data that lies along this eigenvector.
If we look at both the dark current and the power spectrum of the first quadrant again we find the $1/f$ noise is greatly reduced by approximately a factor of $10\times$.

\begin{center}
\includegraphics[width=4.5in]{hw3-q0-darkcurrent-filtered-subtracted.png}
\end{center}

\begin{center}
\includegraphics[width=4in]{hw3-q0-ps-filtered-subtracted.png}
\end{center}


\section{Source Code}
\lstinputlisting[language=Python,breaklines=true]{../hw3.py}



\end{document}